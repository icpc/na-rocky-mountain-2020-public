\problemname{Train Boarding}

Punctual City is well known for the punctuality of its citizens and
its public transportation system.  It is particularly famous for its
train system.  It is always on time, and never too late (or even too
early).  Statistics about train boarding is regularly collected to
keep things running smoothly.

A train has cars numbered $1$ to $N$ (from front to back), each of
length $L$ meters. Each car has exactly one door for boarding located
at the center ($L/2$ meters from each end of the car). There are no gaps
between cars.

When the train stops at the boarding platform, each passenger waiting
for the train walks to the door of the car which is closest to them,
taking the higher numbered car in the case of a tie.

Given the location of the passengers relative to the train, help the
city by reporting the longest distance that any passenger has
to walk and the maximum number of passengers boarding any single car.

\begin{center}
  \includegraphics[width=0.9\textwidth]{Diagram1.png}
\end{center}

\section*{Input}

The first line of input contains three integers $N$ ($1 \leq N \leq 100$),
which is the number of cars of the train, $L$ ($2 \leq L \leq 100$), which
is the length of each car, and $P$ ($1 \leq P \leq 1\,000$), which is the
number of passengers waiting for the train. It is guaranteed that $L$
is an even number.

The next $P$ lines describe the location of the passengers relative to
the train. Each line contains a single integer $x$
($0 \leq x \leq 10\,000$), which is the distance the passenger is
behind the front-end of the train.

\section*{Output}

Display the longest distance that any passenger has to walk on one
line.  On the next line, display the maximum number of passengers boarding
any single car.
