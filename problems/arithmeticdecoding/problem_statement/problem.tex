\problemname{Arithmetic Decoding}

Arithmetic coding is a method to represent a message as a real number
$x$ such that $0 \leq x < 1$. We will assume that the
message consists only of uppercase `A's and `B's.  The two letters
have associated probabilities $p_A$ and $p_B = 1 - p_A$ such that
$0 < p_A < 1$.

The current interval $[a,b)$ is initially set to $[0,1)$ and we will
update this interval one letter at a time.  To encode
a letter, the current interval is divided into two subintervals
as follows.  Let $c = a + p_A(b-a)$.  If the next letter is `A',
$[a,c)$ becomes the current interval.  Otherwise, the current interval
is now $[c,b)$.  This process is repeated for each letter in the
message.  If $[k,\ell)$ is the final interval, the encoded message is
chosen to be $k$.

For example, if the original message is ``ABAB'' and $p_A = p_B = 0.5$,
the sequence of intervals encountered in the algorithm is
\[ [0,1) \xrightarrow{A} [0, 0.5) \xrightarrow{B} [0.25, 0.5)
  \xrightarrow{A} [0.25, 0.375) \xrightarrow{B} [0.3125, 0.375). \]
The encoded message is therefore 0.3125, or 0.0101 in binary.

Given the length of the message, the probabilities, and the encoded
message, determine the original message.

\section*{Input}

The first line contains the integer $N$ ($1 \leq N \leq 15$), which
is the length of the original message.  The
second line contains the integer $D$ ($1 \leq D \leq 7$),
which indicates that $p_A = \frac{D}{8}$.
The third line contains the binary representation of the encoded message.
It is guaranteed that the binary representation of the encoded message
starts with ``0.'' and contains at most $3N+2$ characters.

It is guaranteed that the encoded message came from an initial message
of length $N$ consisting only of `A' and `B' using this value of $p_A$.

\section*{Output}

Display the original message.
